\documentclass{article}

\usepackage{amsmath, amsfonts, amssymb}
\usepackage{parskip}

\DeclareMathOperator{\sgn}{sgn}


\title{Problem 3c -- Extra Problems: Integration}
\author{Introduction to Engineering Mathematics}
%\date{May 2025}

\begin{document}
\maketitle

For this problem, we need a number of facts that are not described explicitly in the course notes, but that can be derived in a similar way as was done for other inverse trigonometric functions.

\section{Domain and range of the inverse secant}

First, the inverse secant function $\sec^{-1}(x)$ is defined for all $x$ such that $|x| \ge 1$. When $x \ge 1$, we have that $0 \le \sec^{-1}(x) < \pi / 2$, and when $x \le -1$, we have that $\pi/2 < \sec^{-1}(x) \le \pi$. This follows from the definition of $\sec^{-1}(x)$, which we don't cover here.

\section{Derivative of the inverse secant}

Secondly, the derivative of $\sec^{-1}(x)$ can be computed using the chain rule: starting from the cancellation property
\[
    \sec \left( \sec^{-1}(x) \right) = x,
\]
we take the derivative of both sides to obtain
\[
    \frac{d}{dx} \sec^{-1}(x) = \frac{1}{x \tan\left( \sec^{-1}(x) \right)}.
\]
It remains to simplify $\tan \left( \sec^{-1}(x) \right)$. We distinguish two cases: when $x \ge 1$, we have that $0 \le \sec^{-1}(x) < \pi/2$, and therefore
\[
    \tan \left( \sec^{-1}(x) \right) = \sqrt{x^2 - 1} \quad \text{(when $x \ge 1$)}
\]
(you should check this). On the other hand, when $x \le -1$, we have $\pi/2 < \sec^{-1}(x) \le \pi$ and therefore
\[
    \tan \left( \sec^{-1}(x) \right) = - \sqrt{x^2 - 1} \quad \text{(when $x \le -1$)}
\]
(likewise, please check this). These two cases can be summarized into one expression by using the so-called \emph{sign function}:
\[
    \sgn(x) = \begin{cases}
        \phantom{+}1 & \text{when $x > 0$} \\
        -1 & \text{when $x < 0$}
    \end{cases}
\]
so that
\[
\tan \left( \sec^{-1}(x) \right) = \sgn(x) \sqrt{x^2 - 1}
\]
irrespective of $x$.

The derivative of $\sec^{-1}(x)$ therefore becomes
\[
    \frac{d}{dx} \sec^{-1}(x) = \frac{\sgn(x)}{x \sqrt{x^2 - 1}}.
\]

\section{Solving the integral}

We can now solve the problem at hand: by integration by parts, we have that
\begin{align*}
    \int x \sec^{-1}(x) \, dx & = \frac{x^2}{2} \sec^{-1}(x) - \frac{1}{2} \int x^2 \frac{d}{dx} \sec^{-1}(x) \, dx \\
    & = \frac{x^2}{2} \sec^{-1}(x) - \frac{1}{2} \sgn(x) \int \frac{x}{\sqrt{x^2 - 1}} \, dx \\
    & = \frac{x^2}{2} \sec^{-1}(x) -  \frac{1}{2} \sgn(x) \sqrt{x^2 - 1} + C.
\end{align*}

\end{document}