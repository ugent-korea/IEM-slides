\documentclass{article}

\usepackage[a4paper, margin=2cm]{geometry}
\usepackage{amsmath, amsfonts, amssymb}
\usepackage{parskip}
\usepackage{booktabs}
\usepackage{hyperref}

\title{Where to find more information}
\author{Last revision: Monday 25 August 2025}
\date{}

\begin{document}
\maketitle

The course sheet mentions two books where you can read through the material that we cover in this course: Adams and Essex (7th edition and up), and Stewart. Both books are available for consultation and for loan from the IGC library, and also contain a wealth of extra problems. 

In addition, there is a freely available text that covers many of the topics that we also cover (though not all) and that is written by UGent professors: Open Source Calculus (OSC). This book can be downloaded from \url{https://oscalculusandanalysis.ugent.be/}. Choose the version that is listed as ``Calculus Only''.			

\section*{How should you use these resources?}

\textbf{See these books (as well as others in the IGC library) as a source of (many) more problem sets!}

\begin{itemize}
\item For new students: follow the course and take notes. When you review your notes and your problem solutions, read up on the parts that you don't understand, using the table below as a guide. You will find that the books sometimes explain things a little different than in class, or cover some more material: you may skip the extra material.
\item For repeating and for new students: consult the worked problems and the chapter exercises to get more practice. Skip the exercises on topics that we have not covered. In most cases it will be clear which exercises to skip. In doubt, consult with your TA.
\end{itemize}

\section*{Chapter map}

\begin{tabular}{clll}
\toprule
\textbf{Chapter} & \textbf{Topic} & \textbf{Adams \& Essex} & \textbf{Open Source Calculus} \\
\midrule
1 & Trigonometry       & P.7, 3.5              & 5.3--5.4 \\
2 & Coordinate Geometry & P.2, P.3             & -- \\
3 & Limits and Continuity & 1.1--1.4           & 8 \\
4 & Derivatives        & 2.1--2.6, 2.9, 2.11   & 9.1--9.5 \\
5 & Complex Numbers    & A1--A7               & 2.3 \\
6 & Polynomials        & P.6                  & 4.1 \\
7 & Binomial Theorem   & 9.8 (example 1 only) & -- \\
8 & Integration        & 6.1--6.3, 5.4--5.7   & 12.1, 12.4, 13.1 \\
9 & Proof techniques   & --                   & -- \\
\bottomrule
\end{tabular}

Notes:

\begin{itemize}
\item Adams and Essex: does not cover sum/product formulas, trigonometric equations, treats real-valued polynomials in P.6 and complex polynomials in ``The Fundamental Theorem of Algebra''(Appendix II). In general, Adams and Essex goes in much more detail for integration than we cover. You may skip those parts, but they will return in Math 1 and Math 2.		

\item OSC is a little more ``sparse'' in some areas (e.g. it does not cover coordinate geometry in 2D or the binomial theorem, it treats complex numbers only in passing), and more dense in others. In particular, the book covers the epsilon-delta definition for limits from the start. You may skip this, though it comes back in Math 1.

\item Both books go into a little more detail for integration. Unlike in our course, they \emph{define} the integral as the area under the curve, and only later make the link with the antiderivative (through the fundamental theorem of calculus). This is the perspective that Math 1 will use. For now, you can skip this (but you must know how to calculate an integral and its link with antiderivatives and area under the curve, as covered in class).

\end{itemize}
\end{document}